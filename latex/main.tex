\documentclass[a4paper, 11pt]{article}
\usepackage{graphicx}
\usepackage{hyperref}
\usepackage[utf8]{inputenc}

\hypersetup{
    colorlinks=true,
    linkcolor=blue,
    urlcolor=blue,
    citecolor=blue
}

\title{knight_tour_SAT}
\author{wislez.valerian}
\date{November 2025}

\begin{document}
\begin{titlepage}
    \centering
    % --- Logo --- 
    \includegraphics[width=\textwidth]{img/uliege_fsa.png} \\[1 cm]

    % --- Header ---
    \textsc{\Large INFO9015--1 Logic for Computer Science} \\[0.4 cm]

    % --- Title ---
    \rule{\linewidth}{0.2 mm} \\[0.4 cm]
    {\Large \bfseries Project Report --- Knight's Tour SAT solving}
    \rule{\linewidth}{0.2 mm} \\[1.5 cm]

    % --- Minipage ---
    \begin{minipage}[t]{0.49\textwidth}
        \begin{flushleft} \large
            \textbf{Teacher:} \\
                Pascal Fontaine \\[0.5 cm]
            \textbf{Teaching Assistant:} \\
                Tom Clara
        \end{flushleft}
    \end{minipage}
    \hfill 
    \begin{minipage}[t]{0.49\textwidth}
        \begin{flushright} \large
            \textbf{Author:} \\
                Valérian Wislez s200825 
        \end{flushright}
    \end{minipage}
    
    \vfill	
    \textsc{\Large Academic Year 2025--2026}
\end{titlepage}

\section{Introduction}

This paper describes the way each question of the project has been solved. A more comprehensive view on the submitted source code can be found at~\cite{knight_tour_github}.

\section{Naive solution}
The first attempt at solving the problem involves the following set of constraints:

\begin{enumerate}
    \item A cell can only be occupied at one timestep.\footnote{To ensure only one constraint, one can simply ensure ``at most one'' and ``at least one''.}
    \item For a given timestep, the knight can only occupy one cell.
    \item Legal moves of the knight restrict its movement. Formally,
    \[wearere \]
\end{enumerate}

This first version is able to solve 

%The constraints are implemented in \texttt{add\_legal\_moves\_constraints}, \texttt{add\_cell\_constraints\_naive}, \texttt{add\_time\_constraints\_naive}, \texttt{add\_cell\_constraints\_sequential\_counter}, \texttt{add\_time\_constraints\_sequential\_counter}.

\section{Efficient solution}
While the naive solution works well in general, for boards ranging up to \ldots x \ldots it has a quadratic number of constraints for encoding the one cell and one time only.
A more efficient solution would benefit of having less constraints for reducing the search space of the solver.
After doing some research about different encodings~\cite{hoos2007sat} for the ``at-most-one'' family of constraints (refined to ``exactly one'' in this problem), it appeared the naive encoding is actually known as ``pairwise encoding'', and the sequential counter seems to be the one most effective encodings there is.
The sequential counter works by assigning a set of new, auxiliary variables, whose size grows \textit{linearly} with the size of the problem.

This technique can be applied to both the one cell only and one timestep only constraints. 

However, it appears that the sequential counter encoding is not always better than the pairwise encoding, as reported by~\cite{hoos2007sat}
Indeed, because of  

This is implemented in \texttt{}


\section{Counting solutions}
In order to count all solutions for a given board, one simply needs to enumerate all the models found by the SAT solver, for each couple $(i_0, j_0) \in M \times N$. Using the PySAT library, it is easy to do so by calling the \texttt{enum\_models} method after solving the problem and simply count them.

Comparing the results obtained when solving $3 \times 4$ as well as $5 \time 5$ chessboards with the ones showed in~\cite{wikipedia_knights_tour}

This is implemented in \texttt{}

\section{Counting up to symmetry}
The easiest way to count up to symmetry was to enumerate all the solutions, apply symmetries and check whether or not the solution is identical to another solution. 

This is implemented in \texttt{}

\section{Finding additional constraints}



\bibliographystyle{plain}
\bibliography{references}

\end{document}
